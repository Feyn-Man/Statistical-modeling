\chapter{Introduzione}
Quali sono gli indici che mi dicono se ho sviluppato un buon modello? +
adattamento dei dati (fit dei dati) + semplicità del modello (numero dei
parametri). Un modello serve per capire di più riguardo ad un fenomeno e
non per complicarlo (parsimonia).

Qual è la differenza tra modello matematico e statistico?
\(\rightarrow\) Incertezza. Il modello statistico è caratterizzato
dall'errore. L'errore considera la variazione individuale.
\(\rightarrow\) Variabili esplicative che possono essere considerate per
migliorare il modello (quindi diminuire l'errore). \(\rightarrow\) Nel
caso stocastico errori nel rapporto campione-popolazione.

Il lavoro nel costruire un modello statistico sta sia nel diminuire
l'errore ma anche nel capire da dove nasce. 

Come nasce l'elaborazione di un modello statistico?

\begin{itemize}
	\tightlist
	\item
	Teoria, Ovvero formulazione di ipotesi, scoperta di relazion empiriche
	o rapporti di causa effetto tra variabili. Individuazione delle
	variabili esplicative.
	\item
	Dati. Capire quale metodo di raccolta utilizzare in base anche alla
	disponibilità economica che si ha per sviluppare il modello.
	Trattamenti preliminari (pulizia ecc.) e poi tornare al modello.
	Tenere conto dell'eterogeneità dei dati (es. considerando per esempio
	il livello di pericolosità delle acque di un lago, se valutiamo tutte
	le particelle nella loro totalità potremmo non concludere che le acque
	sono pericolose, questo potrebbe infatti risultare valido nella sua
	totalità ma magari identifichiamo delle zone in cui avvengono più
	morti rispetto alla normalità. Questo perché ci potrebbero essere
	delle zone maggiormente inquinate che non emergono da un'analisi
	totale delle acque. Quindi considerare anche campionamenti di questo
	tipo , utilizzare tutti i dati potrebbe non dirci nulla). In questa
	fase rientra anche una prima analisi preliminare dei dati.
	\item
	Specificazione del modello (Probabilistico o descrittivo)
	\item
	Stima dei parametri e verifica dell'adattabilità ai dati
	\item
	Utilizzo
\end{itemize}

Ripetere più volte (se necessario).

\emph{Oss.} Oggi un problema nella costruzione di un modello è anche la
privacy. Ci sono modelli che potrebbero essere molto interessanti ma non
si possono elaborare per problemi di privacy. Quindi devo usare il
modello che ho per correggere i dati in questo senso (Teoria
\(\rightarrow\) Dati). Vale però anche il contrario, ovvero i Dati
aiutano nell costruzione di un modello
(\(\Rightarrow \text{Dati} \rightarrow \text{Teoria}\))

Il modello di base è il \emph{modello di regressione}. La regressione
può essere \emph{semplice}, \emph{multipla} o \emph{multivariata}.
\emph{Semplice}, se si ha una sola variabile dipendente ed una sola
variabile esplicativa. \emph{Multipla}, se si hanno più variabili
esplicative e una sola dipendente. \emph{Multivariata}, se si ha più di
una variabile esplicativa e più di una variabile dipendente.

\textbf{Stima}, ovvero trovare i parametri per il modello. Uno dei
metodi di stima è quello di \emph{regressione lineare}.

\textbf{Verifica} dei risultati sia in termini descrittivi (adattamento
ai dati), poi test statistici sulla significatività. Se la verifica non
conduce ad un rifiuto del modello stimato allora lo si utilizza
altrimenti si torna alla fase di specificazione. 
